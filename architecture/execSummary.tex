\section*{Executive Summary}
\thispagestyle{empty}
\addtolength{\parskip}{\baselineskip}

The Coinvent software will produce novel ideas through unfamiliar combinations of familiar ideas - a difficult computational task. We aim to develop a computationally feasible, cognitively-inspired formal model of concept creation, drawing on Fauconnier and Turner's theory of conceptual blending, and grounding it on a sound mathematical theory of concepts. 

The software must support different domains and different types of user. The key target domains are mathematics and music, but it should be applicable
to a wide range of domains beyond those.

We have mapped out different use cases, and analysed the requirements that stem from them.
By exploring examples in different domains, we have concluded that a high level of flexibility and user control will be required. This design allows for that accordingly. A mathematical case-study where the complex numbers are invented was particularly informative in developing the requirements; this case-study 
has been submitted to the C3GI workshop for publication.

The use-cases have emphasised the importance of the model-generation and evaluation components as key parts of the system, alongside the concept blending engine.

We propose a modular system design, which supports the flexibility necessary to cover the varying domains. The modular design is also a good basis for coordinating the work of the distributed teams involved in the project. 

Recent trends in software development have seen the growth of a RESTful variant of the model-view-controller pattern, where a json-over-http API provides a clear
separation between data, operations, and user-interface. We adopt that pattern here, adding an agent/queue model to support potentially slow computations.

We are fortunate to be building on existing tools which will provide building blocks for several components.

