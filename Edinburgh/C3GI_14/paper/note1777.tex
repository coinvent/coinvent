\documentclass{my-blue-book}
\usepackage{amsfonts}
\usepackage{graphicx}
\usepackage{diagrams}
\usepackage[british]{babel}
\usepackage{hyperref}
% \usepackage{csquotes}
\usepackage[style=authoryear,backend=biber]{biblatex}
\DeclareLanguageMapping{british}{british-apa}
\addbibresource{~/tex/shared-bibtex/papers.bib} %
\urlstyle{tt}   % because some styles use normal font here ...
\newcommand{\nat}{\mathcal{N}}
\newcommand{\real}{\mathcal{R}}
\newcommand{\complex}{\mathcal{C}}
\newcommand{\quaternion}{\mathcal{H}}
\renewcommand{\real}{\mathcal{R}}
\newcommand{\dottriangle}{$.\,\raise1ex\hbox{.}\,.$}                   
\DeclareMathOperator{\plusR}{+_{\real}}
\DeclareMathOperator{\timesR}{\times_{\real}}
\DeclareMathOperator{\plusF}{+_F}
\DeclareMathOperator{\minusF}{-_F}
\DeclareMathOperator{\minusR}{-_{\real}}
\DeclareMathOperator{\timesF}{\times_{F}}
\DeclareMathOperator{\invF}{^{-1_F}}
\DeclareMathOperator{\invR}{^{-1_{\real}}}
\newcommand{\minus}[1]{-{#1}}
\newcommand{\origin}{\mathbf{0}}
\newcommand{\one}{\mathbf{1}}
\newcommand{\oneF}{1_F}
\newcommand{\zeroF}{0_F}
\newcommand{\FnotZero}{F_{\neq 0}}
\newcommand{\mapsTo}{\rightarrow}
\newcommand{\norm}[1]{\lVert#1\rVert}
\newcommand{\modulus}[1]{\lvert#1\rvert}
\newcommand{\realTwo}{\real \times \real}
\newcommand{\foleq}{\mbox{$FOL^=$}\xspace}
\newcommand{\vecU}{\mathbf{u}}
\newcommand{\vecV}{\mathbf{v}}
\newcommand{\iprod}[2]{\langle {#1}, {#2} \rangle}
\DeclareRobustCommand{\VAN}[3]{#2}
\NoteNumber{1777}
\title{Complex numbers as a blend}
\author{Alan Smaill}

\begin{document}
\maketitle

\section{Introduction}

As part of work on the Coinvent project, we are interested in looking
at examples of blends in mathematics;  eventually we want to
be able to look for new, interesting blends, but for the moment 
it's good to have some examples where the details of the
blend are worked out, using the ideas from \textcite{Gog05}.

Here is presented such a set up for the complex numbers.

Some motivation from the cognitive side is in the work of Turner and
Fauconnier, eg in \textcite{Fau02}, who present the complex numbers as a
blend between the two dimensional plane, with its associated geometry,
and the algebra of the real numbers.  The aim in the current note is
not to model the historical development, but to give an example
of a possible construction of the complex numbers as a result
of formation of appropriate theories.

\section{The complex numbers}
\label{sec:complex-numbers}

The complex numbers can be looked at as a blend between the
geometrical notion of normed real vector space, and the algebraic
notion of a field.

As a normed real vector space $V$, we have axioms (in a first-order
sorted presentation, with two sorts, one for vectors and one for
scalars, in this case reals) as follows.

\begin{itemize}
\item The ordered field axioms for $\real$:
  \begin{enumerate}
  \item $\real$ with $\plusR$, real 0  and unary $\minusR$
       form a commutative group;
  \item $\real$ without the real 0,
    with $\timesR$, $\one$ and $\invR$ form a commutative group;
  \item and $\timesR$ distributes over $\plusR$.
  \item $\leq$ is a total order on $\real$, which respects addition,
        and the product of positive numbers is positive.
  \end{enumerate}

\item The elements of $V$ form a commutative group with identity $\origin$
and inverse operation (here as prefix minus).

\begin{alignat}{2}
  \all{x,y:V} \quad&&x + y         &= y + x\\
  \all{x,y,z:V} \quad &&(x + y) + z &= x + (y + x)\\
  \all{x:V}  \quad&&x + \origin    &= x\\
  \all{x:V}  \quad&&x + (\minus{x}) &= \origin
\end{alignat}


\item Interaction between field operations and vector operations.  Here
  the field is the real numbers, operations are written
with subscript $\real$;  $\one$ is the field multiplicative identity.
\begin{alignat}{2}
  \all{x,y:V} \all{\lambda:\real} \quad&& 
            \lambda(x + y) &= \lambda x + \lambda y\\
  \all{x:V} \all{\lambda,\mu : \real} \quad &&
            (\lambda \plusR \mu)x &= \lambda x  + \mu x\\
  \all{x:V} \all{\lambda,\mu : \real} \quad && 
            \lambda(\mu x) &= (\lambda \timesR \mu) x \\
  \all{x:V}  \quad&& \one x &= x
\end{alignat}
\item 

Finally there is a norm on the vector space (giving the size of
vectors): $\norm{.} : V \mapsTo \real$, with associated axioms.

\begin{alignat}{2}
  \all{x:V} \quad && \norm{x} & \geq 0 \\
  \all{x:V} \quad && \norm{x} > 0 & \leftrightarrow x \neq 0 \\
  \all{x:V} \all{\lambda:\real} \quad && \norm{\lambda x} 
                   &= \modulus{\lambda} \timesR \norm{x} \\
  \all{x,y,z:V} \quad && \norm{x + y} & \leq \norm{x} + \norm{y}
\end{alignat}

\end{itemize}

On the other side, we have the field axioms, which are already part
of the vector space axiomatisation.  The desired generalised theory
linking the axiomatisations will not use that link, but it
is an obvious common feature of the two axiomatisations.

For completeness, here is an axiomatisation.

\begin{itemize}
\item $F$ with $\plusF, \zeroF, \minusF$ is a commutative group:
  \begin{alignat}{2}
  \all{x,y:F} \quad&&x \plusF y         &= y \plusF x\\
  \all{x,y,z:F} \quad &&(x \plusF y) \plusF z &= x \plusF (y \plusF x)\\
  \all{x:F}  \quad&&x \plusF \zeroF    &= x\\
  \all{x:F}  \quad&&x \plusF (\minusF{x}) &= \zeroF
\end{alignat}
\item 
  $F$ apart from $\zeroF$ with $\timesF, \oneF, \invF$ is a commutative group.\\
  Here use a subtype $\FnotZero$ of $F$ with an obvious definition.
  \begin{alignat}{2}
  \all{x,y:\FnotZero} \quad&&x \timesF y         &= y \timesF x\\
  \all{x,y,z:\FnotZero} \quad &&(x \timesF y) \timesF z &= x \timesF (y \timesF x)\\
  \all{x:\FnotZero}  \quad&&x \timesF \oneF    &= x\\
  \all{x:\FnotZero}  \quad&&x \timesF x\invF   &= \oneF
\end{alignat}

\item 
 Distributivity:
  \begin{alignat}{2}
  \all{x,y,z:F} \quad&&x \timesF (y \plusF z)  &= 
           (x \timesF y) \plusF (x \timesF z)
\end{alignat}  

\end{itemize}

\section{A couple of generalisations}
\label{sec:couple-gener}

To form a blend between these notions, we look for
a generalised theory which has morphisms into the two theories
above;  let's call these axiomatisations $Ax(NVS)$ and $Ax(F)$
respectively.

There is an obvious morphism whereby $Ax(F)$  is included in $Ax(NVS)$
by a renaming of symbols (i.e.\ by a signature morphism, in this case
not surjective), as follows:
\begin{center}
  $
  \begin{array}[h]{rcl}
   \zeroF & \dashrightarrow & 0\\
   \oneF & \dashrightarrow & \one\\
   \plusF & \dashrightarrow & +\plusR\\
   \timesF & \dashrightarrow & \timesR\\
   \minusF &  \dashrightarrow & \minusR\\
   .\invF  & \dashrightarrow  & .\invR
  \end{array}
  $
\end{center}
Call this inclusion morphism  $Inc$; it maps all the axioms
in $Ax(F)$  to axioms of $Ax(NVS)$. We also have identity morphisms
at our disposal (written as $Id$).

The blend here is simply $Ax(NVS)$, so 
this is an uninteresting notion of blend, since we get nothing new.
The claim here is that the following diagram is a push-out:

\begin{center}
  \begin{diagram}[size=7mm]
    &       &   $Ax(NVS)$   &       & \\
    & \ruTo^{\rotatebox{-45}{$Id$}} &       & \luTo^{\rotatebox{45}{$Inc$}} &          \\
    $Ax(NVS)$ &       &   &       & $Ax(F)$ \\
    & \luTo_{\rotatebox{45}{$Inc$}} &       & \ruTo_{\rotatebox{-45}{$Id$}} &  \\
    & & $Ax(F)$ & &
  \end{diagram}
\end{center}

(This just needs that the inclusion morphism is monic/injective.)

\subsection{A productive generalisation}
\label{sec:prod-gener}

The more interesting generalisation associates \emph{some} of the
field operations with vector operations:
\begin{center}
  $
  \begin{array}[h]{rcl}
   \zeroF & \dashrightarrow & \origin\\
%  \oneF & \dashrightarrow & 1\\
   \plusF & \dashrightarrow & +\\
%  \timesF & \dashrightarrow & \times\\
   \minusF &  \dashrightarrow & -\
%  .\invF  & \dashrightarrow  & .^{-1}
  \end{array}
  $
\end{center}

Now the generalised theory is a sub-theory of the theory of fields,
so that a blend based on this generalisation simply pulls
in the ``missing'' field axioms, while ensuring that the
identifications just made are respected; i.e.\ the axioms
of the blend claim that there is a multiplication defined
over vectors, a unit vector which is the identity for multiplication,
and an inverse operation of vectors such that the field axioms hold.

Such a blend is a \emph{consistent} theory -- for example
the real numbers themselves as a 1-dimensional real vector
space satisfy all the axioms. The literature is a bit
mixed up about what to call this theory --- a normed
commutative unital associative algebra over the reals, perhaps?%
\footnote{\url{http://en.wikipedia.org/wiki/Algebra_over_a_field}}


For present purposes, we want
to see what happens when the dimension of the vector space
is fixed, since then we happen to know that, depending on the
dimension chosen, the full blend may or may not be consistent.

\section{Two dimensions}
\label{sec:two-dimensions}

Looking for syntactic similarities between axiomatisations in this way
is going to be dependent on the exact form of the axiomatisations ---
two equivalent axiomatisations of the same theory may look very
different.  So the choice of representation will matter here.

\subsection{Option 1}
\label{sec:option-1}

One way to go is to take take the two dimensional case as given
without disrupting the axiomatisations above.  The following is
loosely based on \cite{HalmosFDVS}.

Let us take new constants for a basis for a 2-dimensional vector space,
say $\vecU,\vecV$.

Thus we have additional vector space axioms, expressing that:
\begin{itemize}
\item every vector is representable by a linear combination of
these basis vectors;
\item that the basis vectors are linearly
independent;
\item and that they are of unit length.
\end{itemize}
\vspace{-1cm}
\begin{center}
  \begin{alignat}{2}
    \all{x:V} \some{\lambda,\mu:\real} \quad&&
                x  &= \lambda\vecU + \mu\vecV \\
    \all{\lambda,\mu:\real} \quad &&
                \lambda\vecU &+ \mu\vecV  = \origin  \; \imp  \;
                \lambda = \mu = 0 \\
       \quad&& \norm{\vecU}   &=  \norm{\vecV} = \one
  \end{alignat}
\end{center}

This looks good in that we have simply extended the original
axiomatisation. There is therefore an inclusion (signature)
morphism from $Th(NVS)$ to this theory; let's call it $Th(2DNVS)$,
and we can build a push-out on top of this.

On the other hand, a natural but hand-waving step here mathematically
here is to claim that since there is a one-to-one correspondence
between vectors and pairs of reals, we can henceforth work with pairs
of reals.  This is true, but more awkward from a representational
point of view.


\subsection{Option 2}
\label{sec:option-2}
 
An alternative is to give ourselves products of sorts and pairings
directly, and then work with a refinement of $Ax(NVS)$, that is
a one where the axioms $Ax(NVS)$, suitably interpreted, hold, as well
as the two-dimensionality axioms, again suitably interpreted.

So instead of $V$, use $\realTwo$, and define:
\begin{alignat}{2}
  \quad&& (x_1,y_1) + (x_2,y_2) &\eqdef (x_1 \plusR x_2, y_1 \plusR y_2) \\
  \quad&&   \origin            &\eqdef (0,0) \\
  \quad&& -(x,y)            &\eqdef (\minusR x, \minusR y)\\
  \quad&& \lambda(x,y)      &\eqdef (\lambda \timesR x, \lambda \timesR y)\\
  \quad&& \norm{(x,y)}       & \eqdef \sqrt((x \timesR x) \plusR (y \timesR y))
\end{alignat}
Here we make use of real square roots; let's not bother about that for
the moment.  

Now the proof obligation is to show that $Ax(NVS)$ hold.
An alternative way to express this is to say that
the definitions above provide a derived signature morphism from the
axioms based on $\realTwo$ to the theory axiomatised by $Ax(NVS)$.
This is an example of a refinement operation in the sense of 
Extended ML (see eg \textcite{KST95}).

\subsection{Forming a blend}

With either option, we have a new situation with the bottom
of a blend diagram, and again we can look to see if the blend is consistent.
Here $2DVS$ can be either representation above, $Th(Ab)$ is abelian group
theory, $B$ is the blend constructed as a pushout via the
morphisms given as dashed arrows.
\newarrow{Dashto} {}{dash}{}{dash}{>}

\begin{center}
  \begin{diagram}[size=8mm]
    & & &       &   $B$    & &\\
    &  & & \ruDashto(4,2)   & & \luDashto(2,2)  &         \\
$2DVS$    & & &  &           & & & $Ax(F)$ \\
    & \luTo & & &        & \ruTo(2,4)& \\
    & & $Th(NVS)$ &       &  &  & \\
    & & & \luTo &       &  &  \\
    & & & & $Th(Ab)$ & & 
  \end{diagram} 
\end{center}
where we take the new theory to be axiomatised by $Ax(NVF)$ together with
the definitions above (and axioms for the product sorts and pairs).

A way to show that the blend here is consistent is to provide
definitions of multiplication, identity and inverse in terms of what
we already have, satisfying the extra field axioms.  Such
abbreviational definitions are guaranteed to yield a conservative
extension, so not to introduce new inconsistency.

For the case of complex numbers, the standard definitions are at hand
(finding them from scratch is of course a different matter):
\begin{alignat}{2}
  \quad&&   \one          &\eqdef (1,0) \\
  \quad&&   (x_1,y_1) + (x_2,y_2)  &\eqdef 
               ((x_1 \timesR x_2) \plusR (\minusR(y_1 \timesR y_2)) ,
                (x_1 \timesR y_2) \plusR (y_1 \timesR x_2)) \\
  \quad&&   (x,y)^{-1}   & \eqdef 
            \norm{(x,y)}^{-2_{\real}} ( x, \minusR y )
\end{alignat}
(where the inverse square operation is shorthand for the obvious thing).

Now check that the field axioms hold for these definitions.

\section{A stronger starting point?}
\label{sec:strong-start-point}

Although it is natural to associate geometrical notions with
two-dimensional real vector spaces, so far there is nothing that
mentions angles, although we have a notion of norm.  This is surely a
missing ingredient of a rational reconstruction of the discovery
process; we know from Hamilton's own account of the quaternions that
this played a big role in that case.  (It is maybe worth pointing that
that Halmos's account of finite dimensional vector spaces does not
introduce either length or angle at in the initial chapters of his
book on that topic \parencite{HalmosFDVS}.)

An obvious way of bringing in a notion of angle is to take our
vector space to have a notion of an inner product, which in turn
yields the norm we already assumed.  That is, we introduce a function
mapping pairs of vectors $(\vecU,\vecV)$ to a real number
$\iprod{\vecU}{\vecV}$ with the properties:
\begin{alignat}{2}
  \all{x,y:V} \quad && \iprod{x}{y} &= \iprod{y}{x} \\
  \all{x,y:V} \all{\lambda:\real} \quad && \iprod{\lambda x}{y}
                   &= \modulus{\lambda} \timesR \iprod{x}{y} \\
  \all{x,y,z:V} \quad && \iprod{x + y}{z}
                   &= \iprod{x}{z} \plusR \iprod{y}{z} \\
  \all{x:V} \quad && \iprod{x}{x} & \geq 0 \\
  \all{x:V} \quad && \iprod{x}{x}  = 0 & \Rightarrow x = \origin
\end{alignat}
Then the norm of vector $x$ is simply  $\iprod{x}{x}$.

On this point, the following is part of the presentation in 
\cite[ch 2]{Conway03}:
\begin{quote}
  The geometrical properties of the complex numbers follow from the
fact that they form a composition algebra for the Euclidean norm
$$N(x + iy) = x^2 + y^2$$
which means that
$$N(z_1 z_2) = N(z_1)N(z_2).$$

This entails that multiplication by a fixed (non-zero) number $z_0$
multiplies all lengths by $\sqrt{N(z_0)}$; \; \dots \\
\flushright{\cite[p 11]{Conway03}}
\end{quote}

Their claim depends on the fact that, in a real vector space, a Euclidean norm
with this compositional property uniquely determines the corresponding
inner product -- according to Wikipedia, this claim is justified
in \textcite[pp 81--86]{Faraut94}. At any event, this
compositionality for norms played an important role in Hamilton's
discovery of Hamiltonians (see forthcoming note).

\section{Not just axioms \dots}

Given the observation that some properties of mathematical structures
are taken to be key, whether or not they are axioms, suggests
that the blending process should not concentrate entirely on axioms.
While our earlier thoughts on blending focused exclusively on
operations on sets of axioms, it is quite possible to include
other statements such as key theorems in the formalism
associated with the initial theory.

In the case of complex numbers, it is the statement
$$\norm{z_1.z_2}^2 = \norm{z_1}^2.\norm{z_2}^2$$
using a less fussy notation than that above.

Expanding with the usual notation, suppose $z_1 = x_1 + i y_1$
and $z_2 = x_2 + i y_2$ with $z_1.z_2 = x_3 + iy_3$.

We then have:
\begin{eqnarray*}
  x_3 & = x_1.x_2 - y_1.y_2 \\
  y_3 & = x_1.y_2 - x_2.y_1 
\end{eqnarray*}
Now look at the square norms:
\begin{eqnarray*}
  \norm{z_1}^2.\norm{z_2}^2 & = & x_1^2.x_2^2 + x_1^2.y_2^2 + 
                               y_1^2.x_2^2 + y_1^2.y_2^2 \\
 \norm{z_1.z_2}^2  &= &  (x_1.x_2 - y_1.y_2)^2 +  (x_1.y_2 + x_2.y_1)^2 \\
                  &= & x_1^2.x_2^2  + y_1^2.y_2^2 -2 x_1.x_2.y_1.y_2 \\
                  &  &  \quad + x_1^2.y_2^2 + x_2^2.y_1^2 + 2 x_1.x_2.y_1.y_2 
\end{eqnarray*}
                
It is the cancellation between the two terms not involving squares in
the last equation that makes this work. It was the analogous step in
3 dimensions that Hamilton could not make to work.

What we have then is a formula that expresses the product of
(two) sums of two squares as a sum of products of squares.
It was Hamilton's discovery that an analogous result holds
for sums of four squares that convinced him that he had found
something worthwhile.

The full situation for the expressibility of products of sums of square
as sums of squares was not established until much later, in \cite{Hurwitz23}.
Such a formula is only possible in dimensions 1, 2, 4 and 8.  There
is a proof of the result in \cite[pp 67--71]{Conway03}.

\section{Conclusion}
\label{sec:conclusion}

The case of complex numbers already gives us significant number
of choices, and we need more examples to see how the trade-offs work
in giving initial sets of definitions, axioms, and theorems.

The case of Hamiltonians is at least as challenging;  it remains
to be seen how far we can get in looking at the discovery process
in the these examples.

\printbibliography

\end{document}

%%% Local Variables: 
%%% mode: latex
%%% TeX-master: t
%%% End: 
