\documentclass{beamer}


\usepackage{amsmath,amssymb,amsfonts}
\usepackage{hyperref,alltt}
\usepackage[english]{babel}
\usepackage{times}
\usepackage[T1]{fontenc}
\usepackage{graphicx}
%\newcommand{\wand}{\mbox{$-\hspace{-2pt}*$}}
%\newcommand{\med}{\hspace{1cm}}
%\input{defseplogic.tex}
\definecolor{Orange}{RGB}{255,128,0}
\mode<presentation>

  \usetheme{default}
  

  \setbeamercovered{transparent}


\title[] % (optional, use only with long paper titles)
{Using Blending to Generate Analogous Lemmas}



\author[] % (optional, use only with lots of authors)
{Alan Smaill and Ewen Maclean}

\institute[] % (optional, but mostly needed)
{Edinburgh University}

\date[] % (optional)
{6th-7th April 2016}

%\subject{}
% This is only inserted into the PDF information catalog. Can be left
% out.



% If you have a file called "university-logo-filename.xxx", where xxx
% is a graphic format that can be processed by latex or pdflatex,
% resp., then you can add a logo as follows:

 %\pgfdeclareimage[height=0.5cm]{university-logo}{university-logo-filename}
 %\logo{\pgfuseimage{university-logo}}



% Delete this, if you do not want the table of contents to pop up at
% the beginning of each subsection:
%% \AtBeginSubsection[]
%% {
%%   \begin{frame}<beamer>
%%     \frametitle{Outline}
%%     \tableofcontents[currentsection,currentsubsection]
%%   \end{frame}
%% }


% If you wish to uncover everything in a step-wise fashion, uncomment
% the following command:

%\beamerdefaultoverlayspecification{<+->}


\begin{document}

\titlepage


\begin{frame}
\begin{center}{\bf Overview}
\end{center}
\begin{itemize}
\item{What is lemma speculation?}
\item{Why is it necessary?}
\item{Why is it so difficult?}
\item{How can blending help?}
\end{itemize}
\end{frame}

\newpage

\begin{frame}
\begin{center}{\bf Lemma Speculation}\end{center}
\begin{itemize}
\item{Proofs can invoke the "cut rule of inference"}
$$f$$
\item{Exploit other related theorems - known as "lemmas"}
\item{Inventing lemmas requires {\bf creativity}}
\end{itemize}
\end{frame}

\newpage

\begin{frame}
\begin{center}{\bf Lemma Example}\end{center}
Example of generalising factorial to qfact?

\end{frame}

\newpage

\begin{frame}
\begin{center}{\bf Lemma Example (contd.)}\end{center}

\end{frame}

\newpage

\begin{frame}
\begin{center}{\bf Why is it necessary?}\end{center}
induction results in failure

\end{frame}

\newpage

\begin{frame}
\begin{center}{\bf Why is it so difficult?}\end{center}
requires CREATIVITY

\end{frame}

\newpage

\begin{frame}
\begin{center}{\bf How can blending help}\end{center}
\begin{itemize}
\item{Assume human input on one example}
\item{Use blending to model analogy}
\item{Identifying commonalities and computing colimit gives a new theory}
\item{Resulting theory often not consistent - needs generalisation}
\end{itemize}
\end{frame}

\newpage

\begin{frame}
\begin{center}{\bf More complex example}\end{center}
List to tree


\end{frame}


\end{document}



