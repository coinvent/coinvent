\documentclass{beamer}


\usepackage{amsmath,amssymb,amsfonts}
\usepackage{hyperref,alltt}
\usepackage[english]{babel}
\usepackage{times}
\usepackage[T1]{fontenc}
\usepackage{graphicx}
%\newcommand{\wand}{\mbox{$-\hspace{-2pt}*$}}
%\newcommand{\med}{\hspace{1cm}}
%\input{defseplogic.tex}
\definecolor{Orange}{RGB}{255,128,0}
\mode<presentation>

  \usetheme{default}
  

  \setbeamercovered{transparent}


\title[] % (optional, use only with long paper titles)
{Using Blending to Generate Analogous Lemmas}



\author[] % (optional, use only with lots of authors)
{Alan Smaill and Ewen Maclean}

\institute[] % (optional, but mostly needed)
{Edinburgh University}

\date[] % (optional)
{6th-7th April 2016}

%\subject{}
% This is only inserted into the PDF information catalog. Can be left
% out.



% If you have a file called "university-logo-filename.xxx", where xxx
% is a graphic format that can be processed by latex or pdflatex,
% resp., then you can add a logo as follows:

 %\pgfdeclareimage[height=0.5cm]{university-logo}{university-logo-filename}
 %\logo{\pgfuseimage{university-logo}}



% Delete this, if you do not want the table of contents to pop up at
% the beginning of each subsection:
%% \AtBeginSubsection[]
%% {
%%   \begin{frame}<beamer>
%%     \frametitle{Outline}
%%     \tableofcontents[currentsection,currentsubsection]
%%   \end{frame}
%% }


% If you wish to uncover everything in a step-wise fashion, uncomment
% the following command:

%\beamerdefaultoverlayspecification{<+->}


\begin{document}

\titlepage


\begin{frame}
\begin{center}{\bf Overview}
\end{center}
\begin{itemize}
\item{What is lemma speculation?}
\item{Why is it necessary?}
\item{Why is it so difficult?}
\item{How can blending help?}
\end{itemize}
\end{frame}

\newpage

\begin{frame}
\begin{center}{\bf Lemma Speculation}\end{center}
\begin{itemize}
\item{Proofs can invoke the "cut rule of inference"}
\item{Exploit other related theorems - known as "lemmas"}
\item{Inventing lemmas requires {\bf creativity}}
\end{itemize}
\end{frame}

\newpage

\begin{frame}
\begin{center}{\bf Lemma Example}\end{center}
Given two functions for calculating factorial:
\begin{eqnarray*}
fact(0)&=&1\\
fact(s(n))&=&s(n)\times fact(n)\\
qfact(0,n)&=&n\\
qfact(s(n),m)&=&qfact(n,s(n)\times m)
\end{eqnarray*}
Can we show
$$
\forall n:\mathbb{N}.\;fact(n) = qfact(n,1)
$$
\end{frame}

\newpage

\begin{frame}
\begin{center}{\bf Lemma Example (contd.)}\end{center}
No we can't (without horrendous subsequent induction steps)\mbox{}\\
However, we can prove
$$
\forall m,n:\mathbb{N}\;fact(n) \times m = qfact(n,m)
$$
relatively easily. This is a generalisation of the original theorem.
\end{frame}

\newpage

\begin{frame}
\begin{center}{\bf Why is it necessary?}\end{center}
\begin{itemize}
\item induction fails
\item representation change
\item generalisation
\end{itemize}
\end{frame}

\newpage

\begin{frame}
\begin{center}{\bf Why is it so difficult?}\end{center}
requires CREATIVITY
\end{frame}

\newpage

\begin{frame}
\begin{center}{\bf How can blending help}\end{center}
\begin{itemize}
\item{Assume human input on one example}
\item{Use blending to model analogy}
\item{Identifying commonalities and computing colimit gives a new theory}
\item{Resulting theory often not consistent - needs generalisation}
\end{itemize}
\end{frame}

\newpage

\begin{frame}
\begin{center}{\bf Example}\end{center}
\begin{itemize}
\item{Choose a source theorem and source lemma}
\begin{eqnarray*}
\forall n:\mathbb{N}.\;fact(n)&=&qfact(n,1)\\
\forall m,n:\mathbb{N}.\;fact(n) \times m&=&qfact(n,m)\\
\end{eqnarray*}
\item{Identify a target theorem}
\begin{eqnarray*}
\forall n:\mathbb{N}.\;sum(n)&=&qsum(n,0)
\end{eqnarray*}
\end{itemize}
\end{frame}


\newpage

\begin{frame}
\begin{center}{\bf Example contd.}\end{center}
\begin{itemize}
\item use HDTP to identify morphisms:\mbox{}\\
\begin{eqnarray*}
sum&\leftrightarrow&fact\\
qsum&\leftrightarrow&qfact
\end{eqnarray*}
\item{Computing colimit produces target lemma from source lemma}
\item{Requires generalisation of input theories and blend}
\item{Lemma computed in blend}
$$
\forall m,n:\mathbb{N}.\;qsum(n,m) = sum(n)+m
$$
\end{itemize}
\end{frame}



\newpage

\begin{frame}
\begin{center}{\bf More complex example}\end{center}
Imagine a theory of lists with theorem
$$
rev(rev(x)) = x
$$
\noindent which requires a generalisation lemma such as
$$
rev(a @ rev(b)) = b & rev(a)
$$
\end{frame}

\begin{frame}
\begin{center}{\bf More complex example}\end{center}
\includegraphics[width=\textwidth]{ltot.png}
\end{frame}


\end{document}



